% CREATED BY DAVID FRISK, 2016
\chapter{Introduction}
\section{Conventional accelerators}
Conventional accelerators, RF cavities and limited acceleration , beam dump for small to future accelerators, difficulties of beam dumping only getting worse with increasing energy. \\
\section{Plasma wakefield accelerators }
Plasma wakefield accelerators as an alternative to accelerators. History (Tajima/Dawson), how it works, progress/challenges. Recent research also shows that L/PWFA may be a good substitute for the deceleration of the beam as well, regardless of whether the beam is accelerated in wake fields or RF cavities. Describe previous work Tajima's paper (passive), Hanahoe (varying density), Bonatto (active). In this thesis we explore these schemes, and attempt to merge the active and varying density approach in what we call a hybrid scheme. All simulations are carried out with a 1GeV 30pc so-called EuPRAXIA electron beam. 
\section{EuPRAXIA}
EuPRAXIA (\textbf{Eu}ropean \textbf{P}lasma \textbf{R}esearch \textbf{A}ccelerator with e\textbf{X}cellence In \textbf{A}pplications)
\section{Thesis Outline}
This intermediate report details the initial phase of a full-year project on plasma wakefield deceleration and is written in partial fulfilment of the requirements for the degree of Master in Physics. As such, it does not attempt to cover the full scope of the work and research conducted in the first half of this project, but rather aims to provide and introduction to the field, establish the theoretical background and construct the computational framework necessary to perform the research we desire. Having lain the groundwork for the project in this report, the the final-year report will reap the rewards of this work by presenting the full results and outcome of the project.\\
The theory behind plasma wakefield acceleration and plasma beam dumps is covered in section 2. The simulation framework is detailed in section 3, followed by simulation tests and preliminary results in section 4. We conclude this report by summarising the work that has been presented and looking ahead at the work that is to be carried out in the second half of this project.
Sections:\\
- Theory (PWFA, LWFA, energy loss etc.)\\
- Simulation (PIC, EPOCH description, set up framework (“building the experiment”))\\
- Preliminary results (low res results) (wide bunch, varying plasma density, hybrid scheme)(Nothing conclusive just results to show that the simulations have been set up correctly)\\
- Conclusion + looking ahead\\

