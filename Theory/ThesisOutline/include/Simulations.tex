\chapter{Simulations}
\begin{itemize}
\item Ways to simulate plasmas, different techniques/codes etc.
\end{itemize}
\section{Particle-in-Cell Simulations}
\textbf{Why simulations}\\
The highly non-linear features present in the high-energy plasma wakefield phenomena we wish to model do not lend themselves easily to analytical treatments. Fortunately, simulations allow us to study, understand and exploit these phenomena without the need to repeatedly perform expensive and intricate experiments.\\
\textbf{Why use PIC? How Pic works.}\\
The response of the plasma to the propagation of an electron beam could theoretically be simulated by, at time $t_0$, solving Maxwell's equations and calculating the combined electromagnetic fields acting on each particle in the plasma and beam, then by considering each particles velocity could calculate the new positions and velocities of all particles for a small time increase $t_0+\Delta t$. Repeating these computation would in theory lead us to find the approximate plasma response at any arbitrary time $t$. This approach is however computationally impossible if we attempt this approach is however computationally unattainable in most plasma simulations. For instance, if we consider that the plasma in a typical plasma wakefield accelerator [Hanahoe] is on the order of centimetres in extent, with a number density $~10^{20} m^{-3}$, we find that we have on the order of $10^{14}$ electrons in the plasma. All these electrons would have to be included in the simulation and stored with their associated 6-dimensional position and velocity data $(x,y,z,v_x,v_y,v_z)$. Each number would be stored as a 32-bit double precision floating point number, yielding the total data size required for the whole plasma simulation on the order of a petabyte ($10^{15}$ bytes).\\
\indent To circumvent this computational road block we make use of Particle-In-Cell (PIC) codes. The key feature of PIC codes is to represent large collections of physical microscopic particles as smaller collections of macroscopic pseudo-particles, where each macroscopic particle carries the total charge and mass of the microscopic particles it represents. The behaviour of these macro particles is then calculated and used as a representation of the response of the actual plasma. 
\subsection{EPOCH}
Is a second order (?) relativistic P.I.C. code\\
1D,2D or 3D options.\\
The simulations presented in this thesis are generated using the open-source plasma physics PIC simulation code \textsc{EPOCH}, which is based upon the particle push and field update
algorithms of the SRC code [ref]. [compare with fig3.1 hanahoe thesis]\\
Starting from EM fields $\vec{E}_{(n)},\vec{B}_{(n)}$ and charge current $\vec{J}_{(n)}$ present at iteration $n$ [at a specific position, middle of Yee grid?] we obtained the fields at the next time step $n+1$ by computing the resulting fields and currents at an intermediate half-way step $n+1/2$. We do this by first computing the change in the electric field, using Ampere's law, $\Delta \vec{E}_{(n)}$ which we add to our current field such that
\begin{equation}
\vec{E}_{(n+1/2)}=\vec{E}_{(n)}+\frac{\Delta t}{2}\left(c^2\mathbf{\nabla}\times \vec{B}_{(n)}-\frac{\vec{J}_{(n)}}{\epsilon_0}\right)
\end{equation}
from this the magnetic field is given by
\begin{equation}
\vec{B}_{(n+1/2)}=\vec{B}_{(n)}-\frac{\Delta t}{2}\left(c^2\mathbf{\nabla}\times \vec{E}_{(n+1/2)}\right)
\end{equation}
(at which point the particle pusher, detailed below, updates the current to $\vec{J}_{(n+1)}$)\\
at which point we need to update the current to $\vec{J}_{(n+1)}$ in order to proceed finding the fields at time step $n+1$. This is done using the particle pusher. We update the position of each particle 
\begin{equation}
\vec{x}_{(n+1/2)}=\vec{x}_{(n)}+\frac{\Delta t}{2}\vec{v}_{(n)}
\end{equation}
from which we also obtain the intermediate velocity $\vec{v}_{(n)}$ [correct?]. Using the Lorentz force law we then compute the force $\vec{F}_{(n)}=\Delta p/\Delta t$ which gives the momentum at $n+1$ as
\begin{equation}
\vec{p}_{(n+1)}=\vec{p}_{(n)}+q\Delta t\left[\vec{E}_{(n+1/2)}\left(\vec{x}_{(n+1/2)}\right)+\vec{x}_{(n+1/2)}\times\vec{B}_{(n+1/2)}\left(\vec{x}_{(n+1/2)}\right) \right]
\end{equation}
where, the electric fields are extrapolated (?) to the intermediate point $n+1/2$. Then, using $\vec{p}=\gamma m \vec{v}$, we can find the velocity at $n+1$, from which we then have the current $\vec{J}_{(n+1)}$. We then reverse the order of computing such that the magnetic field is calculated prior to the electric field,
\begin{align}
&\vec{B}_{(n+1)}=\vec{B}_{(n+1/2)}-\frac{\Delta t}{2}\left(c^2\mathbf{\nabla}\times \vec{E}_{(n+1/2)}\right)\\
&\vec{E}_{(n+1)}=\vec{E}_{(n+1/2)}+\frac{\Delta t}{2}\left(c^2\mathbf{\nabla}\times \vec{B}_{(n+1)}-\frac{\vec{J}_{(n+1)}}{\epsilon_0}\right)
\end{align}
Using these fields when then calculate the new particles positions $\vec{x}_{(n)}$, we "push" the particles, thus completing the iteration step.
\subsection{Grid settings}
An important feature of PIC codes is the grid parameters. When setting up the resolution of the grid one has to make sure that the grid is sufficiently fine such that the smallest features of our physical system are resolved. This is to ensure that the simulation accurately models the physical system it is meant to represent, to the extent that missing small scale phenomena might alter the large scale outcome of the simulation. A finer grid however requires more macroparticles to fully populate the grid, which inevitably extents the computational time. In addition the time step $\Delta t$ aneeds to be suitably decreased as well. This is because of the so-called Courant-Friedrichs-Lewy (CFL) condition. 
Any simulation introduces uncertainties in the final outcome due to the finite resolution. We need to make sure that the uncertainties introduces during each iteration do not build up and grow unbounded. \\
\vfill
Parameters and initial conditions are defined using an \textit{input.deck} file.
\begin{verbatim}
begin:boundaries
  bc_x_min = simple_laser
  bc_x_max = simple_outflow
  bc_y_min = simple_outflow
  bc_y_max = simple_outflow
end:boundaries
\end{verbatim}
$$n_b = \frac{1}{{\sigma_x\sigma_y^2 (2\pi)^{3/2 }  }}e^{{{ - \left( {x - x_0 } \right)^2 } \mathord{\left/ {\vphantom {{ - \left( {x - x_0 } \right)^2 } {2\sigma_x ^2 }}} \right. \kern-\nulldelimiterspace} {2\sigma_x^2 }}}e^{{{ - \left( {y - y_0} \right)^2 } \mathord{\left/ {\vphantom {{ - \left( {y- y_0 } \right)^2 } {2\sigma_y ^2 }}} \right. \kern-\nulldelimiterspace} {2\sigma_y ^2 }}}e^{{{ - \left( {y - y_0 } \right)^2 } \mathord{\left/ {\vphantom {{ - \left( {y - y_0 } \right)^2 } {2\sigma ^2 }}} \right. \kern-\nulldelimiterspace} {2\sigma_y ^2 }}}$$
\section{Notes.}
Meeting Guoxing:
\begin{itemize}
\item We will change $\sigma_{x,y}$, in simulation from $\sigma_{x,y}=0.3 \mu m ~\to~5-10 \mu m$ because the $0.3\mu m$ EuPRAXIA beam parameter gives to high beam density $n_b$, which means that we can't have $n_b\sim n_p$ because the plasma density would have to be too high. We should aim for $n_p\sim 10^{17}-10^{18}\sim n_b$ (standard L/PWFA) parameters. 
EuPRAXIA wants $\sigma_{x,y}$ small because small bunches gives more coherent radiation in undulators. One could expand the beam by letting it propagate freely (expand due to space charge) a distance before reaching the beam dump. 
\item $\text{Run simulations with uniform plasma density for }\left\{\begin{aligned}
&n_p\sim 0.1 n_b \quad &&\text{Non-linear}\\
&n_p\sim  n_b\quad &&\text{Quasi-linear}\\
&n_p\sim 10 n_b\quad &&\text{Linear}
\end{aligned}\right.$ \\
\item Use $\Delta E/E=0.01$ and bunch charge $30~$pC ($5~$fs).\\
\item Estimate necessary simulation propagation length by saturation length using wave-breaking electric field gradient 
$$L_{\text{sat}}\approx \frac{T_0}{eE_{wb}}=\frac{T_0}{e}\frac{e}{m_e c\omega_p}=\frac{T_0}{m_e c}\sqrt{\frac{m_e e\epsilon_0}{e^2n_b}} $$ 
\item Project outline:
\begin{itemize}
\item Uniform plasma with varying $n_b\sim n_p$
\item Vary plasma density profile
\item Test laser to dump head of beam
\item Run simulations for real FlashForward parameters and not the idealized EuPRAXIA parameters.
\end{itemize}
\item  100pC $$n_b=\frac{N_p}{(2\pi)^{3/2} \sigma_y^2\sigma_x}=\frac{6.25\times 10^{8}}{(2\pi)^{3/2} (5\times 10^{-6})^3}\approx 3.2\times 10^{23}~ \text{m}^{-3} $$
$$\Rightarrow ~~eE_{\text{wb}}=\left\{\begin{aligned}
&17 ~\text{GeV/m }&& n_p=0.1 n_b \\
&54 ~\text{GeV/m} && n_p=n_b\\
&172 ~\text{GeV/m} &&n_p=10 n_b
\end{aligned}\right.\quad\Rightarrow ~~L_{sat}(1 ~\text{GeV})=\left\{\begin{aligned}
&5.8 ~\text{cm}&& n_p=0.1 n_b \\
&1.9 ~\text{cm} && n_p=n_b\\
&0.6 ~\text{cm} &&n_p=10 n_b
\end{aligned}\right.$$
$$1 ~\text{GeV beam} ~~\Rightarrow ~~ L_{sat}\sim 2 ~\text{cm}=2*10^4 \mu \text{m} $$
\item  30pC $$n_b=\frac{N_p}{(2\pi)^{3/2} \sigma_y^2\sigma_x}=\frac{1.87\times 10^{8}}{(2\pi)^{3/2} (5\times 10^{-6})^3}\approx 9.5\times 10^{22}~ \text{m}^{-3} $$
$$\Rightarrow ~~eE_{\text{wb}}=\left\{\begin{aligned}
&9.4 ~\text{GeV/m }&& n_p=0.1 n_b \\
&30 ~\text{GeV/m} && n_p=n_b\\
&94 ~\text{GeV/m} &&n_p=10 n_b
\end{aligned}\right.\quad\Rightarrow ~~L_{sat}(1 ~\text{GeV})=\left\{\begin{aligned}
&10.7 ~\text{cm}&& n_p=0.1 n_b \\
&3.4 ~\text{cm} && n_p=n_b\\
&1.1 ~\text{cm} &&n_p=10 n_b
\end{aligned}\right.$$
$$1 ~\text{GeV beam} ~~\Rightarrow ~~ L_{sat}\sim 3.4 ~\text{cm}=3.4*10^4 \mu \text{m} $$


\end{itemize}