\chapter{Theory}


\section{PWFA - Linear-Fluid Wakefield Theory}
\subsection{Density perturbations}
Perturbation due to beam $n\left(r,\xi \right)\to n\left(r,\xi \right)+n_1\left(r,\xi \right)$, use Maxwell's equations and continuity equation.
\begin{equation}
-\frac{1}{k_p^2}\left(\frac{\partial^2 }{\partial \xi^2}+k_p^2\right)n_1\left(r,\xi \right)=n_b\left(r,\xi \right) ~,~~n_1\left(r,\xi<0 \right)=0
\end{equation}
\begin{equation}
\mathcal{L}_{\xi}n_1\left(r,\xi \right)=n_b\left(r,\xi \right) \quad \Rightarrow \quad \mathcal{L}_{\xi}G\left(\xi,\xi'\right)=\delta\left(\xi-\xi'\right)
\end{equation}
\begin{equation}
G\left(\xi,\xi'\right)=\left\{ \begin{array}{ll}
0 &,~ -\infty<\xi<\xi'\\
A\sin\left(k_p\xi \right) + B\cos\left(k_p\xi \right) &,~ \xi'<\xi<\infty
\end{array}\right.
\end{equation}
where the Green's function obeys the same b.c as the density perturbation, i.e it is continuous across the boundary with a discontinuous derivative across the boundary.
Integrate across discontinuity at $\xi=0$
\begin{equation}
\lim_{\epsilon\to 0}\int_{\xi'-\epsilon}^{\xi'+\epsilon} \mathcal{L}_{\xi}G\left(\xi,\xi'\right)\mathrm{d}\xi=\lim_{\epsilon\to 0}\int_{\xi'-\epsilon}^{\xi'+\epsilon}\delta\left(\xi\right)\mathrm{d}\xi=1 \quad \Rightarrow \quad \lim_{\epsilon\to 0}\left[-\frac{1}{k_p^2}\frac{\partial G}{\partial \xi}\right]^{\xi'+\epsilon}_{\xi'-\epsilon}=1
\end{equation}
For simplicity set arrival of beam at $t=0$, such that $\xi'=0$.
\begin{equation}
G\left(\xi,\xi'\right)=-k_p\sin\left(k_p\xi \right)\Theta\left(\xi \right) \quad \Rightarrow \quad n_1\left(r,\xi \right)=\int_{-\infty}^{\infty}G\left(\xi,\xi'\right)n_b\left(r,\xi' \right) \mathrm{d}\xi'
\end{equation}
\subsection{Longitudinal Accelerating Field} %Accelerating and Focusing Electric Fields
\begin{align}
&\boldsymbol{\nabla}\times \vec{E}=-\frac{1}{c}\frac{\partial \vec{B}}{\partial t} \\
&\boldsymbol{\nabla}\times \vec{B}=\frac{4\pi}{c}\vec{J}+\frac{1}{c}\frac{\partial \vec{B}}{\partial t}
\end{align}
gives
\begin{equation}
\nabla^2\vec{E}-\frac{1}{c^2}\frac{\partial^2 \vec{E}}{\partial t^2}=\frac{4\pi}{c^2}\frac{\partial \vec{J}}{\partial t}+4\pi\boldsymbol{\nabla}\rho
\end{equation}
Lorentz for law ($\vec{v}$ is the velocity of the plasma):
\begin{equation}
m\frac{\partial n\vec{v}}{\partial t}=en\left(\vec{E}+\frac{\vec{v}\times\vec{B}}{c} \right)\approx en\vec{E} \quad \Rightarrow \quad
\frac{\partial \vec{J}_p}{\partial t}=\frac{e^2 n}{m}\vec{E}
\end{equation}
Letting $\rho=\rho_b+\rho_p$ and $\vec{J}=\vec{J}_b+\vec{J}_p$ for the beam and plasma respectively, and $\vec{J}_b=c\rho_b\hat{\vec{z}}$, gives
\begin{equation}
\left(\nabla^2-\frac{1}{c^2}\frac{\partial^2}{\partial t^2}-k_p^2\right)\vec{E}=\frac{4\pi}{c}\frac{\partial \rho_b}{\partial t}\hat{\vec{z}}+4\pi\boldsymbol{\nabla}\left(\rho_b+\rho_p\right)
\end{equation}
where $k_p=\omega_p/c$ is the plasma wave number. To find the electric field along the beam, z-direction, we proceed by solving
\begin{equation}
\left(\nabla^2-\frac{1}{c^2}\frac{\partial^2}{\partial t^2}-k_p^2\right)E_z=\frac{4\pi}{c}\frac{\partial \rho_b}{\partial t}+4\pi\frac{\partial}{\partial z}\left(\rho_b+\rho_p\right)
\end{equation}
using $\nabla^2=\nabla^2_{\perp}+\partial^2_z$, in Fourier transform space, where
\begin{equation}
\wtilde{E}_z\left(k\right)=\int_ {-\infty}^{\infty}E_z(\xi)e^{-ik\xi}\mathrm{d}k~~,~~~\wtilde{\rho}_b\left(\xi\right)+\wtilde{\rho}_p\left(\xi\right)=\int_ {-\infty}^{\infty}\left(\rho_b(k)+\rho_b(k)\right)e^{ik\xi}\mathrm{d}k
\end{equation}
such that 
\begin{equation}
\left(\frac{\partial^2 }{\partial z^2}-\frac{1}{c^2}\frac{\partial^2 }{\partial t^2}\right)\wtilde{E}_z\left(\xi\right)=0
\end{equation}
which gives 
\begin{equation}
\left(\nabla^2_{\perp}-k^2_p\right)\wtilde{E}_z\left(\xi\right)=\frac{4\pi}{c}\frac{\partial \wtilde{\rho}_b}{\partial t}+4\pi\frac{\partial }{\partial z}\left(\wtilde{\rho}_b+\wtilde{\rho}_p\right)
\end{equation}
From eq. XXX we have
\begin{equation}
\frac{\partial^2{\rho}_p}{\partial t^2}+\omega_p^2{\rho}_p=-\omega_p^2{\rho}_b \quad \Rightarrow\quad
-k^2\wtilde{\rho}_p+k_p^2\wtilde{\rho}_p=-k_p^2\wtilde{\rho}_b \quad \Rightarrow\quad \wtilde{\rho}_p=\frac{k_p^2}{k^2-k_p^2}\wtilde{\rho}_b
\end{equation}

\begin{equation}
\nabla_{\perp}^2=\frac{1}{r}\frac{\partial }{\partial r}r\frac{\partial }{\partial r} +\frac{1}{r^2}\frac{\partial^2 }{\partial \phi^2} 
\end{equation}
\begin{equation}
 \left(\frac{\partial^2 }{\partial r^2}+\frac{1}{r}\frac{\partial }{\partial r} -k^2_p\right)\wtilde{E}_z
=\left[\frac{4\pi}{c}\frac{\partial}{\partial t}+4\pi\left(1+\frac{k_p^2}{k^2-k_p^2}\right)\frac{\partial }{\partial z}\right]\wtilde{\rho}_b
\end{equation}
We now rewrite this equation as
\begin{equation}
\mathscr{L}\wtilde{E}_z= \wtilde{f}(r)
\end{equation}
We proceed as before and solve this PDE by finding the Green's function. We assume that the source is radially symmetric such that $\mathscr{L}G(r,r')=\delta(r-r')$. The LHS of eq. XXX is the modified Bessel function of order zero and the RHS represents our source term. Consequently the Green's function is formed by linear combinations of  modified Bessels functions of order zero.
\begin{equation}
G\left(r,r'\right)=\left\{ \begin{array}{ll}
A(r_0)(A_1 I_0(r)+B_1K_0(r)) &,~ 0<r<r'\\
B(r_0)(A_2 I_0(r)+B_2K_0(r))  &,~ r'<r<\infty
\end{array}\right.
\end{equation}


\newpage
Now lets solve this for the a beam charge distribution that is a delta function in $z$, yet with radial symmetry
\begin{equation}
\rho_b=\frac{e}{2\pi r}\delta(r-r_0)\delta(\xi) \quad \Rightarrow\quad  \left(r^2\frac{\partial^2 }{\partial r^2}+r\frac{\partial }{\partial r} -r^2k^2_p\right)\wtilde{E}_z=2ei\frac{kk_p^2}{k^2-k^2_p}r\delta(r-r_0)
\end{equation}
We may compute the electric field from the Green's function directly, or by first computing the point-particle and then convolving it with the point-field, which ever is more computationally demanding. 
