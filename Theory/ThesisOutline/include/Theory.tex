% CREATED BY DAVID FRISK, 2016
\chapter{Theory}


\section{PWFA - Linear Regime}
Perturbation due to beam $n\left(r,\xi \right)\to n\left(r,\xi \right)+\tilde{n}\left(r,\xi \right)$, use Maxwell's equations and continuity equation.
\begin{equation}
-\frac{1}{k_p^2}\left(\frac{\partial^2 }{\partial \xi^2}+k_p^2\right)\tilde{n}\left(r,\xi \right)=n_b\left(r,\xi \right) ~,~~\tilde{n}\left(r,\xi<0 \right)=0
\end{equation}
\begin{equation}
\mathcal{L}_{\xi}\tilde{n}\left(r,\xi \right)=n_b\left(r,\xi \right) \quad \Rightarrow \quad \mathcal{L}_{\xi}G\left(\xi,\xi'\right)=\delta\left(\xi\right)
\end{equation}
\begin{equation}
G\left(\xi,\xi'\right)=\left\{ \begin{array}{ll}
0,~~ & -\infty<\xi<0\\
A\sin\left((k_p\xi \right) + B\cos\left(k_p\xi \right),~~ & 0<\xi<\infty
\end{array}\right.
\end{equation}
where the Green's function obeys the same b.c as the density perturbation, i.e it is continuous across the boundary with a discontinuous derivative across the boundary.
Integrate across discontinuity at $\xi=0$
\begin{equation}
\lim_{\epsilon\to 0}\int_{-\epsilon}^{\epsilon} \mathcal{L}_{\xi}G\left(\xi,\xi'\right)\mathrm{d}\xi=\lim_{\epsilon\to 0}\int_{-\epsilon}^{\epsilon}\delta\left(\xi\right)\mathrm{d}\xi=1 \quad \Rightarrow \quad \lim_{\epsilon\to 0}\left[-\frac{1}{k_p^2}\frac{\partial G}{\partial \xi}\right]^{\epsilon}_{-\epsilon}=1
\end{equation}
\begin{equation}
G\left(\xi,\xi'\right)=-k_p\sin\left(k_p\xi \right)\Theta\left(\xi \right) \quad \Rightarrow \quad \tilde{n}\left(r,\xi \right)=\int_{-\infty}^{\infty}G\left(\xi,\xi'\right)n_b\left(r,\xi' \right) \mathrm{d}\xi'
\end{equation}
\section{Equation}


\section{Table}
\begin{table}[H]
\centering
\caption{Values of $f(t)$ for $t=0,1,\dots 5$.}
\begin{tabular}{l|llllll} \hline\hline
$t$ & 0 & 1 & 2 & 3 & 4 & 5 \\ \hline
$f(t)$ & 1 & 1 & 4 & 9 & 16 & 25 \\ \hline\hline
\end{tabular}
\end{table}

\section{Chemical structure}
\begin{center}
\chemfig{X*5(-E-T-A-L-)}
\end{center}

\section{List}
\begin{enumerate}
  \item The first item
  \begin{enumerate}
    \item Nested item 1
    \item Nested item 2
  \end{enumerate}
  \item The second item
  \item The third item 
  \item \dots
\end{enumerate}

\section{Source code listing}
%\lstset{language=Matlab}
\begin{lstlisting}[frame=single]
% Generate x- and y-nodes
x=linspace(0,1); y=linspace(0,1);

% Calculate z=f(x,y)
for i=1:length(x)
 for j=1:length(y)
  z(i,j)=x(i)+2*y(j);
 end
end
\end{lstlisting}

\section{To-do note}
The \texttt{todo} package enables to-do notes to be added in the page margin. This can be a very convenient way of making notes in the document during the process of writing. All notes can be hidden by using the option \emph{disable} when loading the package in the settings. \todo{Example of a to-do note.}

