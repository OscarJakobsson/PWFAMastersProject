\chapter{Ongoing work and concluding remarks}
The natural continuation of the simulation results presented in figure \ref{quasiplots}(a) is to conduct the same theoretical comparisons as described in chapter \ref{theorychapter}. Preliminary analysis shows that the linear fluid is still predictive in the quasilinear regime. For consistency, however, we choose to postpone the full theoretical analysis of this situation until the derived energy-loss equation (\ref{energy_loss_bonatto}) has been properly implemented. It is furthermore desirable to include the space-charge force in our theoretical analysis to compare against the transverse wakefield, thus estimating the density compression of the bunch as it travels through the plasma. \\
\indent The modifications to EPOCH described in section \ref{Nonanalytical}, to allow a laser pulse to be introduced in front of the bunch in figure \ref{quasiplots}(a), have been successfully tested. The laser pulses we have tested so far display rapid dispersion in the plasma, to the extent that we have not been able to demonstrate any laser-driven deceleration. To this end, we are now working in close collaboration with Alexandre Bonatto to implement a so-called plasma channel in our EPOCH simulations. This channel is composed of a non-uniform plasma density which increases parabolically in the radial direction . Studies have shown that by optimizing the shape of such a channel dispersionless laser propagation can be achieved in 3D simulations. We have however not been able to replicate this result in our 2D simulations. Current work is concerned with investigating why this is and how to otherwise proceed with the active beam dump simulations, given that we are at the present computationally limited to 2D simulations.

