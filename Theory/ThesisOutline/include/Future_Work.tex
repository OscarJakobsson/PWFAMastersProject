\chapter{Ongoing work and concluding remarks}
The natural continuation of the simulation results presented in figure \ref{quasiplots}(a) is to conduct the same theoretical comparisons as described in chapter \ref{theorychapter}. Preliminary analysis shows that the linear fluid is still predictive in the quasilinear regime. For consistency, however, we choose to postpone the full theoretical analysis of this situation until the derived energy-loss equation (\ref{energy_loss_bonatto}) has been properly implemented. \\
\indent The modifications to EPOCH described in section \ref{Nonanalytical}, to allow a laser pulse to be introduced in front of the bunch in figure \ref{quasiplots}(a), have been successfully tested. The laser pulses we have tested so far however display rapid dispersion in the plasma, to the extent that we have not been able to demonstrate any laser-driven deceleration. To this end, we are now working in close collaboration with Alexandre Bonatto to implement a so-called plasma channel in our EPOCH simulations. This channel is composed of a non-uniform plasma density which increases parabolically in the radial direction . Studies have shown that by optimizing the shape of such a channel dispersionless laser propagation can be achieved in 3D simulations \cite{Bonatto2015, PlasmaChannel}. We have however not been able to replicate this result in our 2D simulations. Current work is concerned with investigating why this is and how to otherwise proceed with the active beam dump simulations, given that we are at the present computationally limited to 2D simulations.\\
\indent If these issues can be resolved, the theoretical background and computational framework presented in this report will allow us to explore the hybrid scheme for the first time and make further progress towards a compact plasma beam dump. 
\vfill 
\subsection*{Acknowledgements}
The author expresses gratitude to Yangmei Li for her support and advice throughout this project, Alexandre Bonatto for his continued support with our active beam dump studies, Kieran Hanahoe for notifying us about the numerical Cherenkov instability and Guoxing Xia for providing excellent supervision.  