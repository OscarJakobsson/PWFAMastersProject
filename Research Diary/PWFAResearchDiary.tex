
\documentclass[%
onecolumn, notitlepage,
%superscriptaddress,
%groupedaddress,
%unsortedaddress,
%runinaddress,
%frontmatterverbose, 
%preprint,
%showpacs,preprintnumbers,
%nofootinbib,
%nobibnotes,
%bibnotes,
 amsmath,amssymb,
 aps,
%pra,
%prb,
%rmp,
%prstab,
%prstper,
%floatfix,
]{article}
\usepackage[
top    =2.5cm,
bottom = 2.5cm,
left   = 1.5cm,
right  = 1.5cm]{geometry}
%\usepackage{amsmath}
%\usepackage[tbtags]{amsmath}
\usepackage{graphicx}% Include figure files
\usepackage{dcolumn}% Align table columns on decimal point
\usepackage{bm}% bold math
\usepackage{mathtools}
\usepackage{physics}
\usepackage{natbib}
\usepackage{varwidth}
\usepackage{amsmath}
\usepackage{amssymb}		
\usepackage{fancyhdr}
\usepackage{float}
\usepackage[usenames, dvipsnames]{color}
\usepackage{tikz} 
\usepackage{ltxgrid}
\usetikzlibrary{shapes,arrows,positioning,automata,backgrounds,calc,er,patterns}
\usepackage{tikz-feynman}
\tikzfeynmanset{compat=1.0.0}
\usetikzlibrary{decorations.shapes}
\tikzset{decorate sep/.style 2 args=
{decorate,decoration={shape backgrounds,shape=circle,shape size=#1,shape sep=#2}}}
%\usepackage{tikz,newtxmath}
\usetikzlibrary{arrows,decorations.markings}
\usepackage[pdfencoding=auto]{hyperref}
\usepackage[utf8x]{inputenc} 
\usepackage{accents}

\usepackage{moreverb}		
\usepackage{listings}	

\usepackage[T1]{fontenc}
\usepackage{lmodern}
\usepackage{textcomp}
%\usepackage{pdflscape}
\usepackage{rotating}
\usepackage{color}
\usepackage{enumitem, xcolor}
\definecolor{MancPurple}{RGB}{93,42,122}
\definecolor{MancYellow}{RGB}{252,207,32}

\newcommand*{\dt}[1]{%
  \accentset{\mbox{\large\bfseries .}}{#1}}

\definecolor{testcolour}{rgb}{0.149019,0,0.439215}
\hypersetup{colorlinks=true,citecolor=testcolour ,urlcolor=testcolour,linkcolor=testcolour,filecolor=testcolour}
\pagestyle{fancy}
\fancyhf{}
%\fancyhead[LE,RO]{Share\LaTeX}
\fancyhead[RE,LO]{\textsc{MpPhys $|$ PWFA}}
\fancyfoot[CE,CO]{\leftmark}
\fancyfoot[LE,RO]{\thepage}
 \newcommand*{\Scale}[2][4]{\scalebox{#1}{$#2$}}
\renewcommand{\headrulewidth}{2pt}
\renewcommand{\footrulewidth}{1pt}
\DeclareMathOperator{\sinc}{sinc}
\newcommand{\fvec}[1]{\underaccent{\Scale[1]{\sim}}{#1}}
\fancypagestyle{firstpage}{%
  \fancyhf{}
\fancyhead[LE,RO]{ \textsc{OSCAR JAKOBSSON $|$ 2018}}
\fancyhead[RE,LO]{\textsc{MPhys $|$ PWFA}}
\fancyhead[CE,CO]{\textsc{RESEARCH DIARY}}
\fancyfoot[LE,RO]{\thepage}
}
\usepackage{accents}
 \DeclareMathAccent{\wtilde}{\mathord}{largesymbols}{"65}
\usepackage{hyperref}

\newenvironment{tightcenter}{%
  \setlength\topsep{0pt}
  \setlength\parskip{0pt}
  \begin{center}
}{%
  \end{center}
}

\def\Xint#1{\mathchoice
   {\XXint\displaystyle\textstyle{#1}}%
   {\XXint\textstyle\scriptstyle{#1}}%
   {\XXint\scriptstyle\scriptscriptstyle{#1}}%
   {\XXint\scriptscriptstyle\scriptscriptstyle{#1}}%
   \!\int}
\def\XXint#1#2#3{{\setbox0=\hbox{$#1{#2#3}{\int}$}
     \vcenter{\hbox{$#2#3$}}\kern-.5\wd0}}
\def\ddashint{\Xint=}
\def\dashint{\Xint-}

\let\oldhat\hat
\renewcommand{\hat}[1]{\oldhat{{#1}}}
\renewcommand{\vec}[1]{\mathbf{#1}}


\def\dbar{{\mathchar'26\mkern-12mu \mathrm{d}}} 

\begin{document}

%\preprint{APS/123-QED}
\thispagestyle{firstpage}
\begin{minipage}{0.7\textwidth}
\vspace{-5pt}
\noindent \textbf{Project:} A compact plasma beam dump for next generation particle accelerators.\\
\noindent \textbf{Supervisor:} Dr. Guoxing Xia.\\ 
\noindent \textbf{Duration:} 24 Weeks, 2018-19.\vspace{-19pt}\\
\begin{tabbing}
\textbf{Location:}  \=The University of Manchester,\\
\>Cockcroft Accelerator Group,\\
\>Manchester, UK.
\end{tabbing}
\end{minipage}


\begin{figure}[h]
\vspace{-95pt}\hspace{0.85\textwidth}
\includegraphics[scale=0.4]{ManchesterCrest.pdf} 
\end{figure}\vspace{-35pt}
\noindent\rule{0.72\textwidth}{0.4pt}\\
\\
\noindent \textbf{Week 1}
\begin{itemize}
\item[\textcolor{MancPurple}{\textbullet}]  First meeting with Guoxing: Discussed project outline, necessary background reading and the \texttt{EPOCH} software. Received documents from Guoxing: LFWA PhD thesis \citep{Chou2016}, PWFA and beam dump papers \citep{Bonatto2015,Bonatto2016,Lu2005,Wu2010,Chou2016a,Hanahoe2017} and  \texttt{EPOCH} users manual \citep{Bennett2015}.
\item[\textcolor{MancPurple}{\textbullet}] Read theory section in Hanahoe's thesis \citep{Hanahoe2017}.
\item[\textcolor{MancPurple}{\textbullet}] \underline{Weekly outline:} Background reading to understand the theory of LWFA, PWFA and plasma wakefield deceleration. I will study the beam dump so no laser will feature in my simulations, however good to study LWFA to understand more about PWFA. There are several PIC softwares: QuickPIC, OSIRIS, VLPL3D, Vorpal, EPOCH, XOOPIC, OOPIC, LCODE...why use EPOCH?
\item[\textcolor{MancPurple}{\textbullet}] \underline{Concepts covered:}
 \begin{itemize}
\item[\textcolor{MancPurple}{\textopenbullet}] Gessner (2.3-2.4) - \textbf{Linear regime}: Response of a cold, non-interacting plasma, with a ultra-relativistic "delta function" particle beam by considering the electron density perturbation $n_1$. Compute Green's function for $n_1$ and then convolve the solution with a 2-D Gaussian beam to get the plasma's response of a non-point-like beam, (2.18 Gessner). I could produce 2D plot $n_1/n_0$ to show this response.
\item[\textcolor{MancPurple}{\textopenbullet}] Derive wave equations for $\vec{E}$ and $\vec{B}$ fields in the plasma following the density perturbation.
\item[\textcolor{MancPurple}{\textopenbullet}] Derive longitudinal and transverse $E$-fields for the ultra-relativistic delta function driving bunch, then convolve with extended Gaussian beam.
\item[\textcolor{MancPurple}{\textopenbullet}] \textbf{Non-linear regime}: Derive accelrating and transverse fields in the blow-out regime (Gessner 2.7).
\end{itemize}
\item[\textcolor{MancPurple}{\textbullet}] I read Wu et al.  - \textit{Collective deceleration: Toward a compact beam dump}\citep{Wu2010} . No yet summarised the theory section in this paper.


\item[\textcolor{MancPurple}{\textbullet}] \underline{Questions:} 
\begin{itemize}
\item[\textcolor{MancPurple}{\textopenbullet}] Should I look at LWFA theory as well, even though we won't have lasers present in the beam dump? Actually, the active scheme proposed by Bonatto et al. is laser-driven so I should look at LWFA as well, right?
\item[\textcolor{MancPurple}{\textopenbullet}] Should I also look at positron beam dumping if I am to look at the ILC beam dump?
\item[\textcolor{MancPurple}{\textopenbullet}] How does the transverse electric field vary with $\phi$ if we choose a beam that is not radially symmetric? 
\end{itemize}


\item[\textcolor{MancPurple}{\textbullet}] \underline{Concepts to look up:} 
\begin{itemize}
\item[\textcolor{MancPurple}{\textopenbullet}] Bump-on-tail instability.
\item[\textcolor{MancPurple}{\textopenbullet}] Landau damping
\item[\textcolor{MancPurple}{\textopenbullet}] Plasma beatatron-wavelength
\item[\textcolor{MancPurple}{\textopenbullet}]  Look up current beam dumps for e-colliders and tabeltop LWFAs .
\end{itemize}


\end{itemize}
\textbf{Linear regime:}
Perturbation due to beam $n\left(r,\xi \right)\to n\left(r,\xi \right)+\tilde{n}\left(r,\xi \right)$, use Maxwell's equations and continuity equation.\\
\textit{-- Density:}
\begin{equation}
-\frac{1}{k_p^2}\left(\frac{\partial^2 }{\partial \xi^2}+k_p^2\right)\tilde{n}\left(r,\xi \right)=n_b\left(r,\xi \right) ~,~~\tilde{n}\left(r,\xi<0 \right)=0
\end{equation}
\begin{equation}
\mathcal{L}_{\xi}\tilde{n}\left(r,\xi \right)=n_b\left(r,\xi \right) \quad \Rightarrow \quad \mathcal{L}_{\xi}G\left(\xi,\xi'\right)=\delta\left(\xi\right)
\end{equation}
\begin{equation}
G\left(\xi,\xi'\right)=\left\{ \begin{aligned}
&0&&, -\infty<\xi<0\\
&A\sin\left((k_p\xi \right) + B\cos\left(k_p\xi \right)&&, 0<\xi<\infty
\end{aligned}\right.
\end{equation}
where the Green's function obeys the same b.c as the density perturbation, i.e it is continuous across the boundary with a discontinuous derivative across the boundary.
Integrate across discontinuity at $\xi=0$
\begin{equation}
\lim_{\epsilon\to 0}\int_{-\epsilon}^{\epsilon} \mathcal{L}_{\xi}G\left(\xi,\xi'\right)\mathrm{d}\xi=\lim_{\epsilon\to 0}\int_{-\epsilon}^{\epsilon}\delta\left(\xi\right)\mathrm{d}\xi=1 \quad \Rightarrow \quad \lim_{\epsilon\to 0}\left[-\frac{1}{k_p^2}\frac{\partial G}{\partial \xi}\right]^{\epsilon}_{-\epsilon}=1
\end{equation}
\begin{equation}
G\left(\xi,\xi'\right)=-k_p\sin\left(k_p\xi \right)\Theta\left(\xi \right) \quad \Rightarrow \quad \tilde{n}\left(r,\xi \right)=\int_{-\infty}^{\infty}G\left(\xi,\xi'\right)n_b\left(r,\xi' \right) \mathrm{d}\xi'
\end{equation}
\noindent \textbf{Week 2}\\
\clearpage
 \vfill
 \bibliographystyle{unsrt}
 \bibliography{PWFA.bib}


 \end{document}
 